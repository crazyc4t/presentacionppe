\documentclass{beamer}
\usepackage[utf8]{inputenc}
\usepackage[spanish]{babel}

\setbeamertemplate{navigation symbols}{}
\usetheme{Berlin}

\title{Ingeniería en sistemas}
\author{Said Neder}
\subtitle{Escrito en \LaTeX}
\institute[\href{https://www.soler.edu.ec/}{Soler Lux Dei}]{Soler Lux Dei\\ \href{https://github.com/crazyc4t/presentacionppe}{\tiny https://github.com/crazyc4t/presentacionppe}}

\begin{document}

\maketitle

\section{Motivos}

\begin{frame}
\frametitle{Whoami}

\begin{itemize}
\item ¿Porqué me interesa la ingeniería en sistemas?
\item ¿Que voy a hacer como ingeniero en sistemas?
\item ¿Porqué es importante mi profesión?
\item ¿Que haré para cumplir mi objetivo?
\end{itemize}

\end{frame}

\section{Objetivos}

\begin{frame}
\frametitle{Mi proyecto}

Nuestros derechos y la mejora de 'InfoSec' en Guayaquil.\\

\begin{itemize}
\item ¿Qué es 'InfoSec'?
\item ¿Porqué debo mejorar mi 'InfoSec'?
\item ¿Cómo hacer conciencia sobre este tema?
\end{itemize}

\end{frame}

\section{Herramientas}

\begin{frame}
 \frametitle{Herramientas}
 \begin{itemize}
  \item FOSS.
  \item Gestor de contraseñas.
    \begin{itemize}
     \item Keepass
     \item Bitwarden
    \end{itemize}

  \item Alternativas a FANG.
  \begin{itemize}
   \item ProtonMail, Tutanota
   \item Signal, Session
   \item 'Custom ROM'
   \item Libreoffie
   \item GIMP, Krita
   \item 'Homelab'
  \end{itemize}

  \item Conocimiento de seguridad ofensiva.
 \end{itemize}

\end{frame}


\end{document}
