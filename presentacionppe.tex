\documentclass{beamer}
\usepackage[utf8]{inputenc}
\usepackage[spanish]{babel}

\setbeamertemplate{navigation symbols}{}
\usetheme{Berlin}

\title{Ingeniería en sistemas}
\author{Said Neder}
\subtitle{Escrito en \LaTeX}
\institute[\href{https://www.soler.edu.ec/}{Soler Lux Dei}]{Soler Lux Dei\\ \href{https://gist.github.com/crazyc4t/b9d600b489b931cb378a59646cb872b1}{\tiny https://gist.github.com/crazyc4t/b9d600b489b931cb378a59646cb872b1}}

\begin{document}

\maketitle

\section{Whoami}

\begin{frame}
\frametitle{Motivos}

El código de esta presentación es completamete código abierto.

\begin{itemize}
\item ¿Porqué me interesa la ingeniería en sistemas?
\item ¿Que voy a hacer como ingeniero en sistemas?
\item ¿Porqué es importante mi profesión?
\item ¿Que haré para cumplir mi objetivo?
\end{itemize}

\end{frame}

\section{Objetivos}

\begin{frame}
\frametitle{Mi proyecto}

Nuestros derechos y la mejora de 'InfoSec' en Guayaquil.\\

\begin{itemize}
\item ¿Qué es 'InfoSec'?
\item ¿Porqué debo mejorar mi 'InfoSec'?
\item ¿Cómo hacer conciencia sobre este tema?
\end{itemize}

\end{frame}

\end{document}
